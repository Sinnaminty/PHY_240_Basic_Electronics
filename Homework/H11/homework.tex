\documentclass[12pt]{exam}

\usepackage{amsmath}
\usepackage{amssymb}
\usepackage{circuitikz}
\usepackage{pgfplots}
\pgfplotsset{compat=1.18}
\usepackage[lmargin=71pt, tmargin=1.2in]{geometry}  %For centering solution box
\thispagestyle{empty}
\begin{document}

\nopointsinmargin
\pointformat{}

\begingroup
\centering
\LARGE PHY 240: Basic Electronics\\
\LARGE Homework Problem H11\\[0.5em]
\large \today\\
\large Aiden Rivera\par
\endgroup

\rule{\textwidth}{0.4pt}

\printanswers

\begin{questions}
\question \textbf{RC Detective.}\\
Consider this circuit, in which we wish to determine the values of $R$ and $C$.
Unfortunately, $R$ and $C$ are locked up inside of a device (represented by the
dashed line) that prevents their direct measurement.

\begin{center}
\begin{circuitikz}[american voltages]
    \node (vina) at (-3,2) {};
    \node (vinb) at (-3,0)  {};
    \node (vouta) at (4,2) {};
    \node (voutb) at (4,0)  {};

    \node (vin) at (-3,1) {$V_{in}$};
    \node (vout) at (4,1) {$V_{out}$};

    \draw
    (vina) to[R, l=$1\,\text{k}\Omega$, o-] (-1,2)
    to[short, -o] (vouta)

    (vinb) to[short, o-o] (voutb)

    (0,2) to[C, l=$C$] (0,0)
    (2,2) to[R, l=$R$] (2,0)

    (-1,3) to[dashed] (3,3)
    to[dashed] (3, -1)
    to[dashed] (-1, -1)
    to[dashed] (-1, 3)
    ;
\end{circuitikz}
\end{center}

In order to determine the values of $R$ and $C$ within the device, we place a
known resistor of resistance 1 k$\Omega$ in front of the device, hold the input voltage
at 10 V for a long time, and then allow it to drop suddenly to 0 V at time
t = 1 ms. While we do this, we monitor both $V_{in}$ and $V_{out}$. The results are
shown in the plot below (in which the dashed curve represents $V_{in}$ and the
solid curve represents $V_{out}$).

Use the plot to determine $R$ and $C$. Here are a few hints:
\begin{itemize}
  \item You can use the steady state behavior of the circuit to determine $R$!
  \item You can use the capacitor’s discharge curve to determine the product $R_{eff} C$!
  \item You can relate $R$ to $R_{eff}$ by considering how the two resistors are configured from the point of view of a discharging $C$.
\end{itemize}
Graph here.
\newpage

\begin{solution}
  Finding $R$ is easier, so let's do that first.\\
\end{solution}
\end{questions}
\end{document}
