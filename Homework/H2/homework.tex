\documentclass[12pt]{exam}

\usepackage{amsmath}
\usepackage{amssymb}
\usepackage{circuitikz}
\usepackage[lmargin=71pt, tmargin=1.2in]{geometry}  %For centering solution box
\thispagestyle{empty}

\begin{document}

\nopointsinmargin
\pointformat{}

\begingroup
\centering
\LARGE PHY 240: Basic Electronics\\
\LARGE Homework Problem H2\\[0.5em]
\large \today\\
\large Aiden Rivera\par
\endgroup

\rule{\textwidth}{0.4pt}

\printanswers

\begin{questions}

\question Voltages and Currents in Eight Simple Circuits.
\begin{parts}
\part
You are given a battery and three identical light bulbs (incandescent
bulbs). Provide a schematic for a circuit containing these four elements in
which the current through Bulb 1, $I_1$, is twice that through Bulb 2, $I_2$.
Neither of these currents can be zero. Use appropriate schematic symbols
for your circuit elements, and be sure to label your bulbs 1, 2, and 3

\part
Now provide a schematic for a circuit containing these four elements in
which the voltage across Bulb 1, $V_1$, is twice that across Bulb 2, $V_2$. Neither
of these voltages can be zero. Use appropriate schematic symbols and label
your bulbs 1, 2, and 3.

\part
Redraw your circuit from part (a) with an ammeter included in a config-
uration that will allow the ammeter to measure the current flowing through
Bulb 2. Use the appropriate schematic symbol for an ideal ammeter.

\part
Redraw your circuit from part (b) with a voltmeter included in a config-
uration that will allow the voltmeter to measure the voltage across Bulb 2.
Use the appropriate schematic symbol for an ideal voltmeter.

\part
Draw the schematic for a circuit in which a 10 V battery, a 100 $\Omega$ resistor,
and a 220$\Omega$ resistor are all in series with one another. Determine the voltage
across each resistor and the current flowing through each resistor.

\part
Draw the schematic for a circuit in which a 10 V battery, a 100 $\Omega$ resistor,
and a 220$\Omega$ resistor are all in parallel with one another. Determine the voltage
across each resistor and the current flowing through each resistor.

\part
Draw the schematic for a circuit in which a 100 mA current source, a 100 $\Omega$
resistor, and a 220 $\Omega$ resistor are all in series with one another. Determine
the voltage across each resistor and the current flowing through each resistor.
Use an appropriate schematic symbol for an ideal current source.

\part
Draw the schematic for a circuit in which a 100 mA current source, a
100 $\Omega$ resistor, and a 220 $\Omega$ resistor are all in parallel with one another.
Determine the voltage across each resistor and the current flowing through
each resistor.
\end{parts}
\newpage

\begin{solution}
\begin{parts}

% a) 
\part
\[
I_1=2I_2
\]
\begin{circuitikz}[american voltages]
\draw
(0,0) to[battery] (0,2)
to[bulb, l=$1$] (2,2)
to[bulb, l=$2$] (4,2)
-- (4,-2)
-- (0,-2)
-- (0,0)
(2,0) to[bulb, l=$3$] (4,0)
(2,2) to[short] (2,0);
\end{circuitikz}
\newpage 

% b)
\part
\[
V_1=2V_2
\]
\begin{circuitikz}[american voltages]
\draw
(0,0) to[battery] (0,2)
to[bulb, l=$2$] (2,2)
to[bulb, l=$1$] (4,2)
-- (4,-2)
-- (0,-2)
-- (0,0)
(2,0) to[bulb, l=$3$] (4,0)
(2,2) to[short] (2,0);

\end{circuitikz}
\newpage

% c)
\part
\[
I_1=2I_2
\]
\begin{circuitikz}[american voltages]
\draw
(0,0) to[battery] (0,2)
to[bulb, l=$1$] (2,2)
to[ammeter] (4,2)
to[bulb, l=$2$] (6,2)
-- (6,-2)
-- (0,-2)
-- (0,0)
(4,0) to[bulb, l=$3$] (6,0)
(4,2) to[short] (4,0);
\end{circuitikz}
\newpage

% d)
\part
\[
V_1=2V_2
\]
\begin{circuitikz}[american voltages]
\draw
(0,0) to[battery] (0,2)
to[bulb, l=$2$] (2,2)
to[bulb, l=$1$] (4,2)
-- (4,-2)
-- (0,-2)
-- (0,0)
(2,0) to[bulb, l=$3$] (4,0)
(2,2) to[short] (2,0)
(0,2) to[short] (0,4)
to[voltmeter] (2,4)
-- (2,0)
;
\end{circuitikz}
\newpage

% e)
\part
\[
V=IR
\]
\[
10V=I320\Omega
\]
\[
I=0.03125A
\]
Since this circuit is in series, the current must be constant everywhere.
\[
V_1=(0.03125A)(100\Omega)=3.125V
\]
\[
V_2=(0.03125A)(220\Omega)=6.875V
\]
\begin{circuitikz}[american voltages]
\draw
(0,0) to[V=$10V$] (0,2)
to[R=$R_1 100\Omega$] (2,2)
to[R=$R_2 220\Omega$] (4,2)
-- (4,0)
-- (0,0)
;
\end{circuitikz}
\newpage

% f)
\part
\[
V=IR
\]
\[
10V=I(68.75\Omega)
\]
\[
I\approx0.145A
\]
Since this circuit is in parallel, the voltage must be constant everywhere.
\[
I_1=\frac{10V}{100\Omega} = 0.1A
\]
\[
I_2=\frac{10V}{220\Omega} \approx 0.045A
\]
\begin{circuitikz}[american voltages]
\draw
(0,0) to[V=$10V$] (0,2)
-- (4,2)
to[R=$R_2 220\Omega$] (4,0)
-- (0,0)
(2,2) to[R=$R_1 100\Omega$] (2,0)
;
\end{circuitikz}
\newpage

% g)
\part
\[
V=IR
\]
\[
V=(0.1A)320\Omega
\]
\[
V=32V
\]
Since this circuit is in series, the current must be constant everywhere.
\[
V_1=(0.1A)(100\Omega)=10V
\]
\[
V_2=(0.1A)(220\Omega)=22V
\]
\begin{circuitikz}[american voltages]
\draw
(0,0) to[I=$0.1A$] (0,2)
to[R=$R_1 100\Omega$] (2,2)
to[R=$R_2 220\Omega$] (4,2)
-- (4,0)
-- (0,0)
;
\end{circuitikz}
\newpage

% h)
\part
\[
V=IR
\]
\[
V=0.1A(68.75\Omega)
\]
\[
V=6.875V
\]
Since this circuit is in parallel, the voltage must be constant everywhere.
\[
I_1=\frac{6.875V}{100\Omega} = 0.06875A
\]
\[
I_2=\frac{6.875V}{220\Omega} \approx 0.03A
\]
\begin{circuitikz}[american voltages]
\draw
(0,0) to[I=$0.1A$] (0,2)
-- (4,2)
to[R=$R_2 220\Omega$] (4,0)
-- (0,0)
(2,2) to[R=$R_1 100\Omega$] (2,0)
;
\end{circuitikz}

\end{parts}
\end{solution}
\end{questions}
\end{document}
