\documentclass[12pt]{exam}

\usepackage{amsmath}
\usepackage{amssymb}
\usepackage{circuitikz}
\usepackage[lmargin=71pt, tmargin=1.2in]{geometry}  %For centering solution box
\thispagestyle{empty}

\begin{document}

\nopointsinmargin
\pointformat{}

\begingroup
\centering
\LARGE PHY 240: Basic Electronics\\
\LARGE Homework Problem H4\\[0.5em]
\large \today\\
\large Aiden Rivera\par
\endgroup

\rule{\textwidth}{0.4pt}

\printanswers

\begin{questions}

\question The Voltage Divider.
\begin{parts}
\part
Draw the schematic for a voltage divider that uses your 5 V source and
two resistors to achieve a 3 V output voltage. Indicate the resistances of
your resistors clearly on your schematic, and indicate also where the output
voltage, $V_{out}$, is to be taken.

\part
We demand that our output voltage does not sag by more than 0.2 V
when a load of 1 $k\Omega$ is attached across $V_{out}$. That is, with the load in place,
$V_{out}$ can not drop below 2.8 V. Check whether your circuit in part (a) satisfies
this property. If it does not, redraw the circuit with new resistors in place
that will satisfy this property and give the desired open circuit output voltage
of 3 V. If your circuit from part (a) already satisfies this requirement, simply
redraw it here

\part
In addition to our first two requirements, we also demand that the voltage
divider dissipate less than 100 mW when no load is attached. Draw the
schematic for a voltage divider that satisfies all three of our requirements.
As always, be sure that your resistor values are clearly indicated.

\end{parts}
\newpage

\begin{solution}
\begin{parts}

\part
\[
V_{out} = V_{in} \times \frac{R_2}{R_1 + R_2}
\]
\[
3V = 5V \times \frac{R_2}{1000\Omega + R_2}
\]
\[
3V = 5V \times \frac{1500\Omega}{1000\Omega + 1500\Omega}
\]

\begin{circuitikz}[american voltages]
\draw
  (0,3) to[V=$5V$] (0,0)
  (0,3) -- (2,3)
  to[R=$R_1\, 1000\Omega$] (2,1.5)
  to[R=$R_2\, 1500\Omega$] (2,0)
  (2,1.5) to [short, -*, l=$V_{out+}$] (4,1.5)
  (0,0) -- (2,0)
  (2,0) to [short, -*, l=$V_{out-}$] (4,0)
  ;
\end{circuitikz}

\part
\[
R_{new} = \frac{R_2\times 1000\Omega}{R_2 + 1000\Omega}
\]
\[
2.8V = 5V \times \frac{R_{new}}{1000\Omega + R_{new}}
\]
\[
3V = 5V \times \frac{1500\Omega}{1000\Omega + 1500\Omega}
\]

\begin{circuitikz}[american voltages]
\draw
  (0,3) to[V=$5V$] (0,0)
  (0,3) -- (2,3)
  to[R=$R_1\, 1000\Omega$] (2,1.5)
  to[R=$R_2\, 150\Omega$] (2,0)
  (2,1.5) to [short, -*, l=$V_{out+}$] (4,1.5)
  to[R=$R_L\, 1000\Omega$] (4,0)
  (0,0) -- (2,0)
  (2,0) to [short, -*, l=$V_{out-}$] (4,0)
  ;
\end{circuitikz}
\newpage

\part
\begin{circuitikz}[american voltages]
\draw
  (0,3) to[V=$5V$] (0,0)
  (0,3) -- (2,3)
  to[R=$R_1\, 1000\Omega$] (2,1.5)
  to[R=$R_2\, 5000\Omega$] (2,0)
  (2,1.5) to [short, -*, l=$V_{out+}$] (4,1.5)
  (0,0) -- (2,0)
  (2,0) to [short, -*, l=$V_{out-}$] (4,0)
  ;
\end{circuitikz}

\end{parts}
\end{solution}

\end{questions}
\end{document}
