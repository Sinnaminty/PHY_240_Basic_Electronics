\documentclass[12pt]{exam}

\usepackage{amsmath}
\usepackage{amssymb}
\usepackage{circuitikz}
\usepackage[lmargin=71pt, tmargin=1.2in]{geometry}  %For centering solution box
\thispagestyle{empty}

\begin{document}

\nopointsinmargin
\pointformat{}

\begingroup
\centering
\LARGE PHY 240: Basic Electronics\\
\LARGE Homework Problem H5\\[0.5em]
\large \today\\
\large Aiden Rivera\par
\endgroup

\rule{\textwidth}{0.4pt}

\printanswers

\begin{questions}

\question When Resistors Play Together.
\begin{parts}
\part
When a resistor with resistance $R_1$ is placed in series with a resis-
tor of resistance $R_2$, prove that the total resistance of the combination is
$R_{tot} = R_1 + R_2$. This needs to be an iron-clad proof, with clear steps and
justifications for each step.

\part
When a resistor with resistance $R_1$ is placed in parallel with a resistor
of resistance $R_2$, prove that the total resistance of the combination is $R_{tot} = \frac{R_1 R_2}{R_1 + R_2}$.
Again, this needs to be an iron-clad proof, with clear steps and
justifications for each step.

\part
You are given a black box with three terminals, labeled Rock (R), Paper
(P), and Scissors (S). You know that the box contains five 100 $\Omega$ resistors
connected in a particular way between these terminals. Upon measuring the
resistances between various terminals, you find the following:
\[
  R_{RP} = 300 \Omega
\]
\[
  R_{PS} = 250 \Omega
\]
\[
  R_{RS} = 150 \Omega
\]
Here, $R_{XY}$ represents the resistance between terminals X and Y when the
third terminal is not connected to anything.
Provide a schematic showing how the five resistors are connected within the
box. Be sure to label the box outputs $R$, $P$, and $S$ as appropriate.
\end{parts}
\newpage

\begin{solution}
\begin{parts}
% a) 
\part
We know that $V=IR$ is true, let that be our guiding light. We can say\dots
\[
  V_{total} = IR_{total}
\]
So\dots
\[
  R_{total}=\frac{V_{total}}{I}
\]
By definition:
\[
V_{total}=\sum_{i=1}^{n} V_n
\]
\[
V_{total}=V_1+V_2
\]
But $V=IR$, so naturally\dots
\[
V_{1}=IR_1
\]
\[
V_{2}=IR_2
\]
\[
  R_{total}=\frac{IR_1 + IR_2}{I}
\]
\[
  R_{total}=I\times\frac{R_1 + R_2}{I}
\]
Hence,
\[
  R_{total}=R_1 + R_2 \, \blacksquare
\]

\newpage 
% b)
\part
Once again, $V=IR$, but since these resistors are in parallel, the voltage drop across them are the same, however their current is different.
\[
  V=I_{total}R_{total}
\]
That is to say\dots
\[
  R_{total}=\frac{V}{I_{total}}
\]
Once again, by definition:
\[
I_{total}=\sum_{i=1}^{n} I_n
\]
\[
I_{total}=I_1+I_2
\]
But $I=\frac{V}{R}$, so
\[
  I_1=\frac{V}{R_1}
\]
\[
  I_2=\frac{V}{R_2}
\]
\[
  R_{total}=\frac{V}{\frac{V}{R_1}+\frac{V}{R_2}}
\]
\[
  R_{total}=\frac{V}{V\times(\frac{1}{R_1}+\frac{1}{R_2})}
\]
\[
  R_{total}=\frac{1}{\frac{1}{R_1}+\frac{1}{R_2}}
\]
Now we find a common denominator for this fraction
\[
  R_{total}=\frac{1}{\frac{R_1+R_2}{R_1R_2}}
\]
Hense, 
\[
  R_{total}=\frac{R_1R_2}{R_1+R_2} \, \blacksquare
\]
\newpage
% c)
\part
\[
R_{RP}=300\Omega
\]
\[
R_{PS}=250\Omega
\]
\[
R_{RS}=150\Omega
\]
The simpliest solution is to assume that $R_{RP}$ has 3 resistors in series.
\begin{circuitikz}[american voltages]
    % Define the terminals R, P, S
    \node (R) at (0,3) [label=left:R] {};
    \node (M) at (1.5,3) [] {};
    \node (P) at (4.5,3) [label=right:P] {};
    \node (S) at (3,0) [label=below:S] {};

    % Draw the resistors between the terminals
    \draw
    (R) to[R=$100\,\Omega$, *-*] (1.5,3)
    to[R=$100\,\Omega$] (3,3)
    to[R=$100\,\Omega$, -*] (P)

    (M) to[short] (1.5,2.5)
    (1,2.5) to[short](2,2.5)
    (1,0) to[R=$100\,\Omega$] (1,2.5)
    (2,2.5) to[R=$100\,\Omega$] (2,0)
    (1,0) to[short, -*] (S);

\end{circuitikz}
\end{parts}
\end{solution}
\end{questions}
\end{document}
