\documentclass[12pt]{exam}

\usepackage{amsmath}
\usepackage{amssymb}
\usepackage{circuitikz}
\usepackage[lmargin=71pt, tmargin=1.2in]{geometry}  %For centering solution box
\thispagestyle{empty}

\begin{document}

\nopointsinmargin
\pointformat{}

\begingroup
\centering
\LARGE PHY 240: Basic Electronics\\
\LARGE Homework Problem H10\\[0.5em]
\large \today\\
\large Aiden Rivera\par
\endgroup

\rule{\textwidth}{0.4pt}

\printanswers
\begin{questions}
\question \textbf{Capacitor Review Problem.}\\
\begin{parts}
\part
Draw the schematic diagram for a circuit containing a 10 V DC source,\\
a 1.2 k$\Omega$ resistor, and a 250 nF capacitor all in series.

\part
At a particular instant in time that we shall call $t = 0$, the charge on the
plate of the capacitor facing the positive side of the 10 V supply is 4 $\mu$C. Of
course, this means that the charge on the capacitor plate facing the negative
side of the 10 V supply is -4 $\mu$C. Determine the current flowing through the
resistor at time $t = 0$. Indicate the direction in which this current is flowing
by placing a labeled arrow next to the resistor on your schematic diagram.

\part
If we wait for a very long time (until $t = \infty$), current will cease flowing
in the circuit. After the current has stopped flowing, how much charge is on
the positive plate of the capacitor? Which plate is positive at this point?

\part
Between times $t = 0$ and $t = \infty$, how much charge passes through the
resistor?

\part
How much energy is stored by the capacitor at time $t = 0$? How much is
stored by the capacitor at $t = \infty$?

\end{parts}
\newpage

\begin{solution}
\begin{parts}
% a) 
\part
Get ur schematic here!\\
\begin{center}
\begin{circuitikz}[american voltages]
    \draw
    (0,0) to[battery1, invert, l=10V] (0,2)
    to[R=$1.2\,\text{k}\Omega$] (2,2)
    to[C, l=$250\,\text{nF}$] (4,2)
    -- (4,0)
    -- (0,0)
    ;
\end{circuitikz}
\end{center}
\newpage 

% b)
\part
First, we have to figure out the voltage across the capacitor.
\[
  Q=CV
\]
\[
  V=\frac{Q}{C}
\]
\[
  V=\frac{0.000004\, C}{0.000000250\, F} = 16V
\]
So the voltage across the capacitor is 16 V.\\
Now, we find the voltage across the resistor.
\[
  V_{total} = V_{r} + V_{c}
\]
\[
  V_{r} = V_{total} - V_{c}
\]
\[
  V_{r} = -6V
\]
Calculating the current is trivial.
\[
  I=\frac{V_{r}}{R}
\]
\[
  I=\frac{-6V}{1200\,\Omega}
\]
\[
  I=-0.005A
\]
The current is flowing towards the positive plate.
\begin{center}
\begin{circuitikz}[american voltages]
    \draw
    (0,0) to[battery1, invert, l=10V] (0,2)
    (2,2) to[R, i=$1.2\,\text{k}\Omega$] (0,2)
    (2,2) to[C, l=$250\,\text{nF}$] (4,2)
    -- (4,0)
    -- (0,0)
    ;
\end{circuitikz}
\end{center}
\newpage

% c)
\part
At $t=\infty$ the voltage across the capacitor will equal the supply voltage.
\[
V_c = 10V
\]
\[
Q=CV_c
\]
\[
Q=(0.00000025F)(10V) = 0.000025C
\]
So the positive plate of the capacitor would be at 2.5 $\mu$C.\\
The side facing the positive terminal would be positive.
\newpage

% d)
\part
\[
4\,\mu C - 2.5\,\mu C = 1.5\,\mu C
\]
\newpage

% e)
\part
\[
  E=\frac{1}{2}CV^2
\]
\[
  E_0=\frac{1}{2}(0.00000025F)(16V)^2
\]
\[
  E_0=0.000032 J = 32\,\mu J
\]
So the energy stored in the capacitor at $t=0$ is 32 $\mu J$.
\[
  E_{\infty}=\frac{1}{2}(0.00000025F)(10V)^2
\]
\[
  E_{\infty}=0.0000125 J = 12.5\,\mu J
\]
And the energy stored in the capacitor at $t=\infty$ is 12.5 $\mu J$.
\end{parts}
\end{solution}
\end{questions}
\end{document}
