\documentclass[12pt]{exam}

\usepackage{amsmath}
\usepackage{amssymb}
\usepackage{circuitikz}
\usepackage[lmargin=71pt, tmargin=1.2in]{geometry}  %For centering solution box
\thispagestyle{empty}

\begin{document}

\nopointsinmargin
\pointformat{}

\begingroup
\centering
\LARGE PHY 240: Basic Electronics\\
\LARGE Homework Problem H8\\[0.5em]
\large \today\\
\large Aiden Rivera\par
\endgroup

\rule{\textwidth}{0.4pt}

\printanswers
\begin{questions}
\question \textbf{Real Badderies.}\\
\begin{parts}

\part
A battery with an open circuit terminal voltage of 4 V has a terminal
voltage of 3.7 V when connected to a 100 $\Omega$ load. Determine the internal
resistance of this battery.

\part
The result from part (a) makes us unhappy, as we would like to use a
voltage source that doesn’t sag so much. Suppose we have two of the batteries
from part (a), and use them in parallel to drive the 100 $\Omega$ load. Assuming
that the two batteries are identical, what is the terminal voltage with the load
attached? Has the parallel battery configuration helped to create a “stiffer”
(less saggy) voltage source? Explain clearly how you arrive at your result.

\part
Ok. Last question about batteries. When a particular battery is attached
to a 300 $\Omega$ load, it is found that 19.6 mA flows through the load. When the
300 $\Omega$ load is replaced by a 100 $\Omega$ load, it is found that 56.6 mA flows through
the new load.\\
Does this provide evidence that the battery has an internal resistance? If not,
explain clearly why not. If so, determine the value of the battery’s internal
resistance.
\end{parts}
\newpage

\begin{solution}
\begin{parts}
% a) 
\part
\[
  3.7V=I(100 \Omega\,)
\]
\[
  I=0.037A
\]
\[
  V=I(R_{load} + R_{internal})
\]
\[
  V=IR_{load} + IR_{internal}
\]
\[
  V=V_{load} + IR_{internal}
\]
\[
 IR_{internal} =  V - V_{load} 
\]
\[
  R_{internal} = \frac{V - V_{load}}{I} 
\]
\[
  R_{internal} = \frac{4V - 3.7V}{0.037A} 
\]
\[
  R_{internal} \approx 8.11 \Omega\,
\]
\newpage 
% b)
\part
Batteries in parallel will have the same voltage 4V, but will have a lowered resistance governed by the resistors in parallel equation.\\
\[
  \frac{1}{R_{bats}} = \frac{1}{R_1} + \frac{1}{R_2}
\]
\[
  R_{bats} = \frac{R_1 + R_2}{R_1R_2}
\]
\[
  R_{bats} = \frac {(8.11\Omega\,)(8.11\Omega\,)}{8.11\Omega\, + 8.11\Omega\,}
\]
\[
  R_{bats} \approx 4.06\Omega\,
\]
Now with the resistance of the batteries, we can find the current and voltage through the load.\\
\[
  R_{tot} = R_{load} + R_{bats} = 104.06\Omega\,
\]
\[
  I = \frac{V}{R} = \frac{4V}{104.06\Omega\,} \approx 0.038A
\]
\[
V = IR = (0.038A)(104.06\Omega\,) \approx 3.999V
\]
As we can see, wiring 2 batteries in parallel has created a "stiffer" voltage source by mitigating the internal resistances of the batteries.
\newpage 
% c)
\part
\[
V = IR = (0.0196A)(300\Omega\,) = 5.88V
\]
\[
V = IR = (0.0566A)(100\Omega\,) = 5.66V
\]
Because the current changes on the change of a load, we can assume that this battery has internal resistance.\\
\[
  V = (0.0196A)(300\Omega\, + R_{internal})
\]
\[
  V = (0.0566A)(100\Omega\, + R_{internal})
\]
\[
  0 = (0.0196A)(300\Omega\, + R_{internal}) - (0.0566A)(100\Omega\, + R_{internal})
\]
\[
  0 = (5.88V + (0.0196A)(R_{internal})) - (5.66V + (0.0566A)(R_{internal}))
\]
\[
  0 = 0.22V - (0.037A)(R_{internal}) 
\]
\[
 (0.037A)(R_{internal}) = 0.22V
\]
\[
  R_{internal} = \frac{0.22V}{0.037A} \approx 5.95\Omega\,
\]

\end{parts}
\end{solution}
\end{questions}
\end{document}
