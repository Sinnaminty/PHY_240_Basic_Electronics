\documentclass[12pt]{exam}

\usepackage{amsmath}
\usepackage{amssymb}
\usepackage{circuitikz}
\usepackage[lmargin=71pt, tmargin=1.2in]{geometry}  %For centering solution box
\thispagestyle{empty}

\begin{document}

\nopointsinmargin
\pointformat{}

\begingroup
\centering
\LARGE PHY 240: Basic Electronics\\
\LARGE Homework Problem H6\\[0.5em]
\large \today\\
\large Aiden Rivera\par
\endgroup

\rule{\textwidth}{0.4pt}

\printanswers
\center
\begin{circuitikz}[american voltages]
    \node (a) at (4,2) [label=right:a] {};
    \node (b) at (4,0) [label=right:b] {};

    \draw
    (a) to[short, o-] (2,2)
    to[R,l_=$30\,\Omega$] (0,2)
    to[battery, l_=18V] (0,0)
    -- (2,0)
    to[R,l_=$20\,\Omega$, -o] (b)
    (2,2) to[R=$60\,\Omega$] (2,0)
    ;
\end{circuitikz}
\begin{questions}
\question \textbf{Monsieur Th\'evenin}\\
Consider the circuit above, in which an unkown load may eventually be placed
between the a and b terminals. We know that this circuit may be replaced by
a Th\'evenin Equivalent Circuit consisting of a single voltage source in series
with a single resistance.

\begin{parts}
\part
Determine the Th\'evenin Equivalent Circuit for this circuit and draw it in
the space below. Be sure to label the voltage source and resistance in your
drawing, and include and indicate the terminals a and b.

\part
Demonstrate that your equivalent circuit is correct by doing the following:
\begin{itemize}
    \item Attach a 20 $\Omega$ load between terminals a and b of the original circuit, and determine the current flowing through the load and the voltage.
    \item Attach a 20 $\Omega$ load between terminals a and b of your equivalent circuit, and determine the current flowing through the load and the voltage.
    \item Show that your results for the load current and voltage are the same for these two circuits across the load.
\end{itemize}
\end{parts}
\newpage

\begin{solution}
\begin{parts}
% a) 
\part
Check the open circuit voltage.\\
Let: $R_1 = 30\,\Omega,\, R_2 = 60\,\Omega,\, R_3 = 20\,\Omega$
\[
R_{total}=R_1 + R_2 + R_3 = 110\,\Omega
\]
\[
V_{oc}=V_{source} \times \frac{R_1 + R_2}{R_{total}}
\]
\[
V_{oc}=18V \times \frac{90\,\Omega}{110\,\Omega} \approx 14.727V
\]
Now, we check the short circuit current\dots
\[
  I_{short}=\frac{18V}{90\,\Omega}=0.2A
\]
With $V_{open}=14.727V$ and $I_{short}=0.2A$
\[
R_{th} = \frac{V_{open}}{I_{short}}
\]
\[
R_{th} = \frac{14.727V}{0.2A} \approx 73.636\,\Omega
\]
\begin{circuitikz}[american voltages]
    \node (a) at (2,2) [label=right:a] {};
    \node (b) at (2,0) [label=right:b] {};

    \draw
    (a) to[R, l_=$73.636\,\Omega$, o-] (0,2)
    to[battery, l_=$14.727V$] (0,0)
    to[short, -o] (b)
    ;
\end{circuitikz}
\newpage 
% b)
\part
\begin{itemize}
\item \textbf{Original Circuit}

\begin{circuitikz}[american voltages]
    \node (a) at (4,2) [label=right:a] {};
    \node (b) at (4,0) [label=right:b] {};

    \draw
    (a) to[short, o-] (2,2)
    to[R,l_=$30\,\Omega$] (0,2)
    to[battery, l_=18V] (0,0)
    -- (2,0)
    to[R,l_=$20\,\Omega$, -o] (b)
    (2,2) to[R=$60\,\Omega$] (2,0)
    (a) to[R=$R_{load}\, 20\,\Omega$] (b)
    ;
\end{circuitikz}
\item \textbf{Equivalent Circuit}
\begin{circuitikz}[american voltages]
    \node (a) at (2,2) [label=right:a] {};
    \node (b) at (2,0) [label=right:b] {};

    \draw
    (a) to[R, l_=$73.636\,\Omega$, o-] (0,2)
    to[battery, l_=$14.727V$] (0,0)
    to[short, -o] (b)
    (a) to[R=$R_{load}\, 20\,\Omega$] (b)
    ;
\end{circuitikz}
\end{itemize}
Meowmeowmeow.
\end{parts}
\end{solution}
\end{questions}
\end{document}
