\documentclass[12pt]{exam}

\usepackage{amsmath}
\usepackage{amssymb}
\usepackage{circuitikz}
\usepackage{pgfplots}
\pgfplotsset{compat=1.18}
\usepackage[lmargin=71pt, tmargin=1.2in]{geometry}  %For centering solution box
\thispagestyle{empty}
\begin{document}

\nopointsinmargin
\pointformat{}

\begingroup
\centering
\LARGE PHY 240: Basic Electronics\\
\LARGE Homework Problem H12\\[0.5em]
\large \today\\
\large Aiden Rivera\par
\endgroup

\rule{\textwidth}{0.4pt}

\printanswers

\begin{questions}
\question \textbf{What Does it Do?}\\
Consider this circuit:
\begin{center}

\begin{circuitikz}[american voltages]
    \node (vina) at (-2,2) {};
    \node (vinb) at (-2,0)  {};
    \node (vouta) at (2,2) {};
    \node (voutb) at (2,0)  {};

    \node (vin) at (-2,1) {$V_{in}$};
    \node (vout) at (2,1) {$V_{out}$};

    \draw
    (vina) to[C, l=$20\,\text{nF}$, o-] (0,2)
    to[short, -o] (vouta)

    (vinb) to[short, o-*] (0,0)
    to[short, -o] (voutb)

    (0,2) to[R, l=$470\,\Omega$] (0,0)
    to (0, 0) node[ground] {}
    ;
\end{circuitikz}
\end{center}
\begin{parts}
\part 
Suppose the capacitor is uncharged at time $t=0$ and we send in the following input signal at $V_{in}$:\\
meow.
\part
meow2
\end{parts}
\newpage

\begin{solution}
\begin{parts}

%a 
\part

\newpage

%b
\part

\newpage
\end{parts}
\end{solution}
\end{questions}
\end{document}
