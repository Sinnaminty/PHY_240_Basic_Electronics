\documentclass[12pt]{exam}

\usepackage{amsmath}
\usepackage{amssymb}
%\usepackage{circuit tek} for circuit stuff, prof Elwood said so!!!
\usepackage[lmargin=71pt, tmargin=1.2in]{geometry}  %For centering solution box
\thispagestyle{empty}

\begin{document}

\nopointsinmargin
\pointformat{}

\begingroup
\centering
\LARGE PHY 240: Basic Electronics\\
\LARGE Homework Problem H1\\[0.5em]
\large \today\\
\large Aiden Rivera\par
\endgroup

\rule{\textwidth}{0.4pt}

\printanswers

\begin{questions}

\question How Fast Are Those lil' Buggers Moving?
\begin{parts}

\part
The first step is to determine the number density of free electrons in a
typical wire. I say “free” electrons because most of the electrons in a wire
are bound to their atoms and may not move freely through the wire. We will
assume that our wire is copper, and here are the only things that you may
use while solving this problem:

\begin{itemize}
\item The density of Copper at room temperature (approximately 8.95 g/cm$^3$).
\item The mass of a single Copper atom (63.55 amu)
\item The fact that each copper atom contriubutes \textit{one} free electron.
\item Avagadro's Number.
\end{itemize}

Use this information to determine the number density of free electrons (that
is, the number of free electrons per volume) in Copper. Explain the reasoning
behind each step of your calculation \textit{clearly}.
\\
\part
Suppose that our wire is an AWG 14 gauge wire1 that has a current of
500 mA flowing through it. Determine the average velocity of the electrons
“down the wire” that is necessary to produce this current.
\\
\part
The answer that you obtained for Part (b) above is called the “electron
drift velocity”, because it is the velocity at which electrons drift down the
wire. Even if you turn the electric current off, however, the electrons are still
moving in the wire (they simply lack a tendency to drift in any particular
direction). Because of the Pauli exclusion principle, the electrons in copper
are stacked up so high that the free electrons in the wire have a kinetic energy
of roughly 7 eV.
\\\\
Convert this 7 eV to Joules. Look up the mass of an electron. Dust off your
Physics I, and use the expression for kinetic energy to determine how fast
these free electrons must be moving when the current is off. The velocity
that you so determine is called the Fermi velocity, $v_{ferm}$.
\\
\part
Let us now turn the 500 mA current back on. Compare the drift velocity,
$v_{drift}$, to the Fermi velocity, $v_{ferm}$. What fraction of $v_{ferm}$ is $v_{drift}$?

\end{parts}
\newpage

\begin{solution}
\begin{parts}
\part
To determine the number density of free electrons in copper, we need to calculate how many free electrons exist per unit volume of copper.
Given data:
\begin{itemize}
    \item \textbf{Density of Copper ($\rho$)}: 8.95 g/cm$^3$
    \item \textbf{Atomic mass of Copper ($M$)}: 63.55 amu (atomic mass units)
    \item \textbf{Avogadro's number ($N_A$)}: $6.022 \times 10^{23}$ atoms/mol
\end{itemize}
The molar mass (in grams per mole) of copper is equivalent to its atomic mass in amu. Therefore:
\[
M = 63.55 \ \text{g/mol}
\]
Using the density ($\rho$) and the molar mass ($M$), we can calculate the molar volume of copper:
\[
\text{Molar volume} = \frac{M}{\rho} = \frac{63.55 \ \text{g/mol}}{8.95 \ \text{g/cm}^3} = 7.1 \ \text{cm}^3/\text{mol}
\]
This is the volume occupied by one mole of copper atoms.
We can now determine the number of atoms per cubic centimeter:
\[
\text{Number density of atoms} = \frac{N_A}{\text{Molar volume}} =
\]
\[
\frac{6.022 \times 10^{23} \ \text{atoms/mol}}{7.1 \ \text{cm}^3/\text{mol}} \approx 8.48 \times 10^{22} \ \text{atoms/cm}^3
\]
Since each copper atom contributes one free electron, the number density of free electrons is the same as the number density of copper atoms:
\[
\text{Number density of free electrons} \approx 8.48 \times 10^{22} \ \text{electrons/cm}^3
\]
\textbf{The number density of free electrons in copper is approximately $8.48 \times 10^{22}$ electrons per cubic centimeter. This means that in every cubic centimeter of copper, there are around $8.48 \times 10^{22}$ free electrons available to conduct electricity.}
\newpage
%-----------------------------------------------------------------------%
\part
To determine the average velocity of the electrons, also known as the drift velocity, we can use the following relationship between the current, charge, and drift velocity:
\[
I = n \cdot A \cdot v_d \cdot e
\]
Where the:
\begin{itemize}
  \item \textbf{Current ($I$):} $0.5 \ \text{A}$
\item \textbf{Number density of free electrons ($n$):} $8.48 \times 10^{28} \ \text{electrons/m}^3$
  \item \textbf{Cross-sectional area of the wire ($A$)}
  \item \textbf{Drift velocity ($v_d$)}
\item \textbf{Elementary charge ($e$):} $1.6 \times 10^{-19} \ \text{C}$
\end{itemize}
Using the formula given for the diameter of an AWG wire, the diameter of the wire is:
\[
d = 0.127\text{mm} \cdot 92^{\frac{22}{39}} \approx 1.628\text{mm}
\]
The cross-sectional area $A$ of the wire is given by:
\[
A = \pi \left(\frac{d}{2}\right)^2 = \pi \left(\frac{1.628 \times 10^{-3} \ \text{m}}{2}\right)^2
\]
\[
A \approx 2.08 \times 10^{-6} \ \text{m}^2
\]
Rearrange the equation $I = n \cdot A \cdot v_d \cdot e$ to solve for $v_d$:
\[
v_d = \frac{I}{n \cdot A \cdot e}
\]
Now, substitute the known values into the equation:
\[
v_d = \frac{0.5 \, \text{A}}{(8.48 \times 10^{28} \, \text{electrons/m}^3) \cdot (2.08 \times 10^{-6} \, \text{m}^2) \cdot (1.6 \times 10^{-19} \, \text{C})}
\]
\[
v_d \approx \frac{0.5}{8.48 \times 10^{28} \cdot 2.08 \times 10^{-6} \cdot 1.6 \times 10^{-19}}
\]
\[
  v_d \approx 1.77 \times 10^{-5} \ \text{m/s}
\]
\textbf{The average drift velocity of the electrons down the wire necessary to produce a current of 500 mA is approximately $0.177 \times 10^{-5} \ \text{m/s}$}
\newpage
%-----------------------------------------------------------------------%
\part
\[
7 \ \text{eV} = 1.122 \times 10^{-18} \ \text{J}
\]
The kinetic energy ($K$) of an electron can be given by the classical expression:
\[
K = \frac{1}{2} m_e v_{\text{Fermi}}^2
\]
Where:
\begin{itemize}
\item \textbf{Kinetic energy ($K$):} $1.122 \times 10^{-18} \ \text{J}$
\item \textbf{Mass of an electron $m_e$:} $9.109 \times 10^{-31} \ \text{kg}$
\item \textbf{Fermi velocity $v_{\text{Fermi}}$}
\end{itemize}
Rearrange the kinetic energy equation to solve for $v_{\text{Fermi}}$:
\[
v_{\text{Fermi}} = \sqrt{\frac{2K}{m_e}}
\]
Substituting the known values:
\[
v_{\text{Fermi}} = \sqrt{\frac{2 \cdot 1.122 \times 10^{-18} \ \text{J}}{9.109 \times 10^{-31} \ \text{kg}}}
\]
\[
v_{\text{Fermi}} = \sqrt{\frac{2.244 \times 10^{-18}}{9.109 \times 10^{-31}}}
\]
\[
v_{\text{Fermi}} \approx \sqrt{2.463 \times 10^{12}} \ \text{m/s}
\]
\[
v_{\text{Fermi}} \approx 1.57 \times 10^6 \, \text{m/s}
\]
\textbf{The Fermi velocity ($v_{\text{Fermi}}$) of the electrons in copper, when the current is off, is approximately $1.57 \times 10^6 \, \text{m/s}$.}
\newpage
%-----------------------------------------------------------------------%
\part
Given:
\begin{itemize}
  \item $v_{\text{drift}} \approx 1.77 \times 10^{-5} \ \text{m/s}$
    \item $v_{\text{Fermi}} \approx 1.57 \times 10^6 \ \text{m/s}$
\end{itemize}
The fraction of $v_{Fermi}$ that is $v_{drift}$ is:
\[
\text{Fraction} = \frac{v_{\text{drift}}}{v_{\text{Fermi}}}
\]
\[
  \text{Fraction} = \frac{1.77 \times 10^{-5} \ \text{m/s}}{1.57 \times 10^6 \ \text{m/s}} \approx 1.13 \times 10^{-11}
\]
\textbf{The drift velocity $v_{\text{drift}}$ is approximately $1.13 \times 10^{-11}$ times the Fermi velocity $v_{\text{Fermi}}$}
\end{parts}
\end{solution}

\end{questions}
\end{document}
