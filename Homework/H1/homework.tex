\documentclass[12pt, letterpaper, onecolumn]{exam}

\usepackage{amsmath}
\usepackage{amssymb}
\usepackage[lmargin=71pt, tmargin=1.2in]{geometry}  %For centering solution box
\lhead{Left Header\\}
\rhead{Right Header\\}
\thispagestyle{empty}   %For removing header/footer from page 1

\begin{document}
\nopointsinmargin
\pointformat{}

\begingroup
\centering
\LARGE PHY 240: Basic Electronics\\[0.5em]
\LARGE Homework Problem H1\\[0.5em]
\large \today\\[0.5em]
\large Aiden Rivera\par
\endgroup

\rule{\textwidth}{0.4pt}

\printanswers
\renewcommand{\solutiontitle}{\noindent\textbf{Ans:}\enspace}

\begin{questions}

\question[0] How Fast Are Those lil' Buggers Moving?
\begin{parts}

\part
The first step is to determine the number density of free electrons in a
typical wire. I say “free” electrons because most of the electrons in a wire
are bound to their atoms and may not move freely through the wire. We will
assume that our wire is copper, and here are the only things that you may
use while solving this problem:\newline

\begin{itemize}
\item The density of Copper at room temperature (approximately 8.95 g/cm$^3$).
\item The mass of a single Copper atom (63.55 amu)
\item The fact that each copper atom contriubutes \textit{one} free electron.
\item Avagadro's Number.\newline
\end{itemize}

Use this information to determine the number density of free electrons (that
is, the number of free electrons per volume) in Copper. Explain the reasoning
behind each step of your calculation \textit{clearly}.\newline

\part
Suppose that our wire is an AWG 14 gauge wire1 that has a current of
500 mA flowing through it. Determine the average velocity of the electrons
“down the wire” that is necessary to produce this current.\newline

\part
The answer that you obtained for Part (b) above is called the “electron
drift velocity”, because it is the velocity at which electrons drift down the
wire. Even if you turn the electric current off, however, the electrons are still
moving in the wire (they simply lack a tendency to drift in any particular
direction). Because of the Pauli exclusion principle, the electrons in copper
are stacked up so high that the free electrons in the wire have a kinetic energy
of roughly 7 eV.\newline

Convert this 7 eV to Joules. Look up the mass of an electron. Dust off your
Physics I, and use the expression for kinetic energy to determine how fast
these free electrons must be moving when the current is off. The velocity
that you so determine is called the Fermi velocity, $v_{ferm}$.\newline

\part
Let us now turn the 500 mA current back on. Compare the drift velocity,
$v_{drift}$, to the Fermi velocity, $v_{ferm}$. What fraction of $v_{ferm}$ is $v_{drift}$?\newline
\end{parts}


\begin{solution}
\begin{parts}
\part 
\part 
\part 
\part 
\end{parts}
\end{solution}

\end{questions}
\end{document}
